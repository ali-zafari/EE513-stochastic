\documentclass[12pt, letterpaper]{scrartcl}

\usepackage{fullpage} % Set margins and place page numbers at bottom center
\usepackage[shortlabels]{enumitem} % Use a. in the enumerate
\usepackage{amsmath} % aligned equations
\usepackage{graphicx} % include figure
\usepackage{float} % usage of H for figure float
\usepackage{amssymb} % \varnothing
\usepackage{subfigure}
\usepackage{xcolor} % color in math mode
\DeclareMathOperator*{\argmin}{argmin} % argmin

\begin{document}

% ### Header - start ###
    \begin{center}
    	\hrule
    	\vspace{0.4cm}
    	{\textbf { {\large Homework 3} \\ EE 513 --- Stochastic Systems Theory}}
    \end{center}
    { \textbf{Name:} Ali Zafari \hspace{\fill} \textbf{Student Number:} 800350381 \hspace{\fill} \textbf{Fall 2022} } \newline\hrule
% ### Header - end ###


\paragraph*{Problem 3.1} \hfill\\
\begin{enumerate}[((a))]
    \item \textbf{Mixed}. As the CDF is is not continuous (jump at $x=0$) the random variable is not continuous. And since the CDF has not a staircase shape, the random variable is not discrete.
    \item
    \begin{align*}
        &P(X<-\frac{1}{2})=0\\
        &P(X<0)=0\\
        &P(X\leq0)=0.25\\
        &P(\frac{1}{4}\leq X<1)=P(X<1)-P(X<\frac{1}{4})=0.5-0.3125=0.1875\\
        &P(\frac{1}{4}\leq X\leq1)=P(X\leq1)-P(X<\frac{1}{4})=1-0.3125=0.6875\\
        &P(X>\frac{1}{2})=1-P(X\leq\frac{1}{2})=1-0.375=0.625\\
        &P(X\geq5)=1-P(X<5)=1-1=0\\
        &P(X<5)=P(X\leq5)-P(X==5)=1-0=1
    \end{align*}
\end{enumerate}
\hrule

\paragraph*{Problem 3.2} \hfill\\
\begin{enumerate}[((a))]
    \item 
    \begin{align*}
        Y=g(X)=e^X \qquad \textrm{(monotonic function)}
    \end{align*}
    {\color{blue} For $y<0$}:
    \begin{align*}
        \qquad F_Y(y)=0 \qquad \text{and} \qquad f_Y(y)=0
    \end{align*}
    {\color{blue} For $y\geq 0$}:
    \begin{align*}
        &F_Y(y)=P(Y\leq y)=P(e^X\leq y)=P(x\leq\ln{y})=F_X(\ln{y})\\
        &f_Y(y)=\frac{d}{dy}F_Y(y)=\frac{d}{dy}F_X(\ln{y})=\frac{d\ln{y}}{dy}\frac{d}{dy}F_X(\ln{y})=\frac{1}{y}F_X^\prime(\ln{y})=\frac{1}{y}f_X(\ln{y})
    \end{align*}
    
    \item 
    \begin{align*}
        X\thicksim\mathcal{N}(m, \sigma^2)
    \end{align*}
    {\color{blue} For $y\geq0$}:
    \begin{align*}
        PDF:&\qquad f_Y(y)=\frac{1}{y}f_X(\ln{y})=\frac{1}{y}\frac{1}{\sqrt{2\pi\sigma^2}}e^{-\frac{(\ln y-m)^2}{2\sigma^2}}
    \end{align*}
\end{enumerate}
\hrule

\paragraph*{Problem 3.3} \hfill\\
\begin{enumerate}[((a))]
    \item
    \begin{align*}
        \int^{+\infty}_{-\infty}f_X(x)dx&=1\\
        \int^{1}_{0}cx(1-x)dx&=1\\
        c[\frac{x^2}{2}-\frac{x^3}{3}]\bigg\rvert^{1}_{0}&=1\\
        c&=6
    \end{align*}
    \item
    \begin{align*}
        W=\pi X^2 \qquad \textrm{(monotonic function on [0,1])}
    \end{align*}
    \begin{align*}
        x=\sqrt{\frac{w}{\pi}}
    \end{align*}
    \begin{align*}
        f_W(w)&=\frac{d}{dw}(\sqrt{\frac{w}{\pi}})f_X(\sqrt{\frac{w}{\pi}})\\
    \end{align*}
    \begin{align*}
        f_W(w)=
        \begin{cases} 
            \frac{6}{2\sqrt{\pi w}}\sqrt{\frac{w}{\pi}}(1-\sqrt{\frac{w}{\pi}})=\frac{3}{\pi}(1-\sqrt{\frac{w}{\pi}}) & 0 \leq w \leq \pi\\
            0 & o.w.\\
        \end{cases}
    \end{align*}
    
    \item
    \begin{align*}
        Z=\frac{4}{3}\pi X^3 \qquad \textrm{(monotonic function on [0,1])}
    \end{align*}
    \begin{align*}
        x=\sqrt[3]{\frac{3z}{4\pi}}
    \end{align*}
    \begin{align*}
        f_Z(z)&=\frac{d}{dz}(\sqrt[3]{\frac{3z}{4\pi}})f_X(\sqrt[3]{\frac{3z}{4\pi}})\\
    \end{align*}
    \begin{align*}
        f_Z(z)=
        \begin{cases} 
            2\sqrt[3]{\frac{9}{16\pi^2z}}(1-\sqrt[3]{\frac{3z}{4\pi}}) & 0 \leq z \leq \frac{4}{3}\pi\\
            0 & o.w.\\
        \end{cases}
    \end{align*}
    \item
    \begin{align*}
        Y=X^n \qquad \textrm{(monotonic function on [0,1])}
    \end{align*}
    \begin{align*}
        x=\sqrt[n]{y}
    \end{align*}
    \begin{align*}
        f_Y(y)&=\frac{d}{dy}(\sqrt[n]{y})f_X(\sqrt[n]{y})\\
    \end{align*}
    \begin{align*}
        f_Y(y)=
        \begin{cases} 
            \frac{6}{n}y^{\frac{2-n}{n}}(1-\sqrt[n]{y}) & 0 \leq y \leq 1\\
            0 & o.w.\\
        \end{cases}
    \end{align*}
\end{enumerate}
\hrule
\paragraph*{Problem 3.4} \hfill\\
\begin{enumerate}[((a))]
    \item
    \begin{align*}
        E[X]&=(-1)\frac{1}{9}+(\frac{1}{2})\frac{4}{9}+(2)\frac{4}{9}=1\\
        E[\frac{1}{X}]&=(-1)\frac{1}{9}+(2)\frac{4}{9}+(\frac{1}{2})\frac{4}{9}=1
    \end{align*}
    thus $E[\frac{1}{X}]= \frac{1}{E[X]}$, in this case.
    \item
    \begin{align*}
        E[X]&=\int_1^2x\times1dx=1.5\\
        E[\frac{1}{X}]&=\int_1^2\frac{1}{x}\times1dx=\ln2
    \end{align*}
    thus $E[\frac{1}{X}]\neq \frac{1}{E[X]}$, in general.
\end{enumerate}
\hrule
\paragraph*{Problem 3.5} \hfill\\
\begin{enumerate}[((a))]
    \item
    \begin{align*}
        M_X[jv]&=E[e^{jvX}]\\
        &=\int_{-\infty}^{+\infty}\frac{\alpha}{2}e^{-\alpha|x|}e^{jvx}dx\\
        &=\frac{\alpha}{2}[\int_{-\infty}^{0}e^{(\alpha+jv)x}dx+\int_{0}^{+\infty}e^{(-\alpha+jv)x}dx]\\
        &=\frac{\alpha}{2}[\frac{1}{\alpha+jv}e^{(\alpha+jv)x}\bigg\rvert^{0}_{-\infty}+\frac{1}{-\alpha+jv}e^{(-\alpha+jv)x}\bigg\rvert^{+\infty}_{0}]\\
        &=\frac{\alpha}{2}[\frac{1}{\alpha+jv}+\frac{-1}{-\alpha+jv}]\\
        &=\frac{\alpha}{2}[\frac{1}{\alpha+jv}+\frac{-1}{-\alpha+jv}]\\
        &=\frac{\alpha^2}{\alpha^2-(jv)^2}
    \end{align*}
    \item
    \begin{align*}
        E[X]&=\frac{d}{djv}M_X[jv]\bigg\rvert_{v=0}\\
        &=\frac{d}{djv}(\frac{\alpha^2}{\alpha^2-(jv)^2})\bigg\rvert_{v=0}\\
        &=\frac{2jv\alpha^2}{(\alpha^2-(jv)^2)^2}\bigg\rvert_{v=0}\\
        &=0
    \end{align*}
    \begin{align*}
        E[X^2]&=\frac{d^2}{d(jv)^2}M_X[jv]\bigg\rvert_{v=0}\\
        &=\frac{d}{djv}(\frac{2jv\alpha^2}{(\alpha^2-(jv)^2)^2})\bigg\rvert_{v=0}\\
        &=\frac{2\alpha^2(\alpha^2-(jv)^2)^2-2(\alpha^2-(jv)^2)(-2jv(2jv\alpha^2))}{(\alpha^2-(jv)^2)^4}\bigg\rvert_{v=0}\\
        &=\frac{2\alpha^6}{\alpha^8}\\
        &=\frac{2}{\alpha^2}\quad\longrightarrow\quad Var[X]=E[X^2]-E[X]^2=\frac{2}{\alpha^2}
    \end{align*}
\end{enumerate}
\hrule
\paragraph*{Problem 3.6} \hfill\\
\begin{enumerate}[((a))]
    \item To have quantized values of $X$, i.e. $q(X)$, have same probability we should have the below equality:
    \begin{align*}
        P(X\leq-a)=P(-a<X\leq0)=P(0<X\leq a)=P(X>a)=\frac{1}{4}
    \end{align*}
    so we can write:
    \begin{align*}
        P(X<-a)&=\frac{1}{4}\\
        \int_{-\infty}^{-a}\frac{1}{2}e^{-|x|}dx&=\frac{1}{4}\\
        \frac{1}{2}e^x\bigg\rvert_{-\infty}^{-a}&=\frac{1}{4}\\
        e^{-a}&=\frac{1}{2}\\
        a&=\ln2\\
        a&=0.6931
    \end{align*}
    \item 
    \begin{align*}
        x_1=\argmin_{x_1}\int_{0}^{a}(x-x_1)^2\frac{1}{2}e^{-|x|}dx\\
    \end{align*}
    To minimize, we calculate the derivative w.r.t $x_1$ and find the value which makes it zero:
    \begin{align*}
        \frac{d}{dx_1}\int_{0}^{a}(x-x_1)^2\frac{1}{2}e^{-x}dx&=0\\
        \int_{0}^{a}-2(x-x_1)\frac{1}{2}e^{-x}dx&=0\\
        \int_{0}^{a}xe^{-x}dx-\int_{0}^{a}x_1e^{-x}dx&=0\\
        -e^{-x}(x+1)\bigg\rvert_{0}^{a}-x_1(-e^{-x})\bigg\rvert_{0}^{a}&=0\\
        x_1&=\frac{1-e^{-a}(a+1)}{1-e^{-a}}\\
        a=\ln2 \quad \longrightarrow \quad x_1&=\frac{1-(1/2)(\ln2+1)}{1/2} \quad \longrightarrow \quad x_1=0.3069
    \end{align*}
    \item
    First, we need to know the values of $q(X)$ for interval of $x\in (a,+\infty)$. We use the result of previous part:
    \begin{align*}
        -e^{-x}(x+1)\bigg\rvert_{a}^{+\infty}-x_2(-e^{-x})\bigg\rvert_{a}^{+\infty}&=0\\
        x_2&=\frac{e^{-a}(a+1)}{e^{-a}}\\
        a=\ln2 \quad \longrightarrow \quad x_2&=\frac{(1/2)(\ln2+1)}{1/2}\quad \longrightarrow \quad x_2=1.6931
    \end{align*}
    The distribution of $q(X)$ is symmetric, so to summarize:
    \begin{align*}
        q(x)=
        \begin{cases} 
            -1.6931 & x\leq -\ln2 \\
            -0.3069 & -\ln2 < x \leq 0\\
            0.3069 & 0 < x \leq \ln2\\
            1.6931 &  x > \ln2
        \end{cases}
    \end{align*}
    Let's now evaluate the MSE:
    \begin{align*}
        E[(X-q(X))^2]&=E[X^2]-2E[Xq(X)]+E[q(X)^2]
    \end{align*}
    \begin{align*}
        &E[X^2]=Var[X]+E[X]^2=2+0=2\\
        &E[Xq(X)]=-1.6931(-0.4233)-0.3069(-0.0767)+0.3069(0.0767)+1.6931(0.4233)=1.4803\\
        &E[q(X)^2]=\frac{1}{4}(1.6931^2+0.3069^2+0.3069^2+1.6931^2)=1.4803
    \end{align*}
    \begin{align*}
        \text{\textbf{MSE}:}\qquad E[(X-q(X))^2]&=2-2(1.4803)+1.4803=0.5197
    \end{align*}
\end{enumerate}
\end{document}

